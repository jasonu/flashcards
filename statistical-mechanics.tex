\documentclass[avery5371,grid]{flashcards}

\cardfrontstyle[\large\slshape]{headings}
\cardbackstyle{empty}
\cardfrontfoot{Quantum Statistical Mechanics}

%\usepackage{amsfonts}
\usepackage{amssymb,amsmath}

\begin{document}

\begin{flashcard}[Copyright \& License]{Copyright \copyright \, 2007 Jason Underdown \\
Some rights reserved.}
\vspace*{\stretch{1}}
These flashcards and the accompanying \LaTeX \, source code are licensed
under a Creative Commons Attribution--NonCommercial--ShareAlike 2.5 License.  
For more information, see creativecommons.org.  You can contact the author at:
\begin{center}
\begin{small}\tt jasonu [remove-this] at physics dot utah dot edu\end{small}
\end{center}
\vspace*{\stretch{1}}
\end{flashcard}

\begin{flashcard}[Definition]{spin excess}
\vspace*{\stretch{1}}
Assuming $N$ is even, then we define the \textit{spin excess} by
\begin{equation*}
N_{\uparrow} - N_{\downarrow} = 2s
\end{equation*}
\vspace*{\stretch{1}}
\end{flashcard}

\begin{flashcard}[Formula]{multiplicity function}
\vspace*{\stretch{1}}
\begin{equation*}
g(N,s) = 
\frac{N!}{\left(\frac{1}{2}N+s \right)! \left(\frac{1}{2}N-s \right)!} =
\frac{N!}{N_{\uparrow}! \; N_{\downarrow}!}
\end{equation*}
\vspace*{\stretch{1}}
\end{flashcard}

\begin{flashcard}[Formula]{Stirling's approximation}
\vspace*{\stretch{1}}
\begin{equation*}
N! \approx (2\pi N)^{1/2} N^{N} \exp(-N +(1/12)N + \cdots)
\end{equation*}
\vspace*{\stretch{1}}
\end{flashcard}

\begin{flashcard}[Formula]{approximate multiplicity function}
\vspace*{\stretch{1}}
\begin{equation*}
G(N,s) \approx (2 / \pi N)^{1/2} 2^{N} \exp(-2s^2/N)
\end{equation*}
\vspace*{\stretch{1}}
\end{flashcard}

\begin{flashcard}[Assumption]{fundamental assumption}
\vspace*{\stretch{1}}
The fundamental assumption of statistical mechanics is that in a closed system,
each of its \textit{accessible} states is \textit{equally likely}.
\vspace*{\stretch{1}}
\end{flashcard}

\begin{flashcard}[Definition]{probability of states}
\vspace*{\stretch{1}}
If $s$ is a state of a system, then the probability of that
state is given by:
\begin{equation*}
P(s) = \left\{ \begin{array}{cl}
1/g & \text{if $s$ is an accessible state} \\
0   & \text{otherwise}
\end{array} \right.
\end{equation*}
The sum of the probabilities over all states is unity.
\begin{equation*}
\sum_{s} P(s) = 1
\end{equation*}
\vspace*{\stretch{1}}
\end{flashcard}

\begin{flashcard}[Definition]{expectation\\average value}
\vspace*{\stretch{1}}
Suppose that a system has some physical property $X=X(s)$ when the
system is in state $s$.  The \textit{expected} or \textit{average value} of $X$ is 
defined by:
\begin{equation*}
\left\langle X \right\rangle = \sum_{s} X(s)P(s)
\end{equation*}
\vspace*{\stretch{1}}
\end{flashcard}

\begin{flashcard}[Definition]{entropy}
\vspace*{\stretch{1}}
\begin{equation*}
\sigma(N,U) \equiv \ln g(N,U)
\end{equation*}
\vspace*{\stretch{1}}
\end{flashcard}

\begin{flashcard}[Equation]{condition for thermal equilibrium}
\vspace*{\stretch{1}}
If two systems are in thermal contact, the condition for them
to be in \textit{thermal equilibrium} is the following:
\begin{equation*}
\left(
\dfrac{\partial \sigma_{1}}{\partial U_{1}}
\right)_{\!\!\! N_1} = \left(
\dfrac{\partial \sigma_{2}}{\partial U_{1}}
\right)_{\!\!\! N_2}
\end{equation*}
\vspace*{\stretch{1}}
\end{flashcard}

\begin{flashcard}[Definition]{fundamental temperature\\Kelvin temperature\\Boltzmann constant}
\vspace*{\stretch{1}}
\begin{equation*}
\dfrac{1}{\tau} \equiv
\left( \dfrac{\partial \sigma}{\partial U} \right)_{\!\!\! N}
\end{equation*}
\bigskip
\begin{equation*}
\tau = k_B T
\end{equation*}
\medskip
\begin{equation*}
k_B = 1.381 \times 10^{-23} \text{ J/K}
\end{equation*}
\vspace*{\stretch{1}}
\end{flashcard}

\begin{flashcard}[Definition]{relationship between entropy\\and classical thermodynamic entropy}
\vspace*{\stretch{1}}
\begin{equation*}
\dfrac{1}{T} = \left( \dfrac{\partial S}{\partial U} \right)_{\!\!\! N}
\end{equation*}
\bigskip
\begin{equation*}
S = k_B \sigma
\end{equation*}
\vspace*{\stretch{1}}
\end{flashcard}

\begin{flashcard}[Equation]{multiplicity function for the Hydrogen atom}
\vspace*{\stretch{1}}
The multiplicity function for a Hydrogen atom with
energy $E_n$, is given by
\begin{equation*}
g(n) = \sum_{l=0}^{n-1} (2l+1)  = n^2
\end{equation*}
where $n$ is the principal quantum number, and $l$ is the orbital quantum
number.
\vspace*{\stretch{1}}
\end{flashcard}

\begin{flashcard}[Equation]{multiplicity function for 3D harmonic oscillator}
\vspace*{\stretch{1}}
The multiplicity function for a simple harmonic oscillator with three
degrees of freedom with energy $E_n$ is given by
\begin{equation*}
g(n) = \dfrac{1}{2}(n+1)(n+2)
\end{equation*}
where $n = n_x + n_y + n_z$.
\vspace*{\stretch{1}}
\end{flashcard}

\begin{flashcard}[]{}
\vspace*{\stretch{1}}

\vspace*{\stretch{1}}
\end{flashcard}

\begin{flashcard}[]{}
\vspace*{\stretch{1}}

\vspace*{\stretch{1}}
\end{flashcard}

\begin{flashcard}[]{}
\vspace*{\stretch{1}}

\vspace*{\stretch{1}}
\end{flashcard}

\begin{flashcard}[]{}
\vspace*{\stretch{1}}

\vspace*{\stretch{1}}
\end{flashcard}

\begin{flashcard}[]{}
\vspace*{\stretch{1}}

\vspace*{\stretch{1}}
\end{flashcard}

\begin{flashcard}[]{}
\vspace*{\stretch{1}}

\vspace*{\stretch{1}}
\end{flashcard}

\end{document}

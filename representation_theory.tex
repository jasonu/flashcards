\documentclass[avery5371,grid]{flashcards}

%% Packages
\usepackage{amsmath}
\usepackage{amsfonts}
\usepackage{amssymb}
\usepackage{ccicons}
\usepackage[mathscr]{euscript}
\usepackage{url}


%% Math macros
\newcommand{\N}{\mathbb{N}}
\newcommand{\Z}{\mathbb{Z}}
\newcommand{\Q}{\mathbb{Q}}
\newcommand{\R}{\mathbb{R}}
\newcommand{\C}{\mathbb{C}}
\newcommand{\B}{\mathscr{B}}
\newcommand{\st}{\textrm{ such that }}
\renewcommand{\le}{\leqslant}
\renewcommand{\theta}{\vartheta}
\newcommand{\iso}{\cong}
\newcommand{\abs}[1]{\ensuremath{\left| #1 \right|}}
\newcommand{\set}[2]{\ensuremath{\left\{ #1 \, : \, #2 \right\}}}
\newcommand{\presentation}[2]{\ensuremath{\left< #1 \, : \, #2 \right>}}
\newcommand{\GLnF}[2]{\ensuremath{\textrm{GL} \left( #1 \, , \, #2 \right)}}
\newcommand{\normal}{\ensuremath{\lhd}}
\DeclareMathOperator{\Ker}{\ensuremath{\textrm{Ker}}}
\DeclareMathOperator{\Img}{\ensuremath{\textrm{Im}}}
\DeclareMathOperator{\End}{\ensuremath{\textrm{End}}}

%% Text macros
\newcommand{\defn}[1]{\textbf{#1}}

%% Layout of flash cards
\cardfrontstyle[\large\slshape]{headings}
\cardbackstyle{empty}
\cardfrontfoot{Representation Theory}

\begin{document}

%%%%%%%%%%%%%%%%%%%%%%%%%%%%%%%%%%%%%%%%%%%%%%%%%%%%%%%%%%%%%%%%%%%%%%%%%%%%%%%
\begin{flashcard}[Copying]
  { Flash Cards for the Book:

    \begin{center}
      ``Representations and Characters of Groups'' \\
      by Gordon James and Martin Liebeck
    \end{center}
  }
  \vspace*{\stretch{1}}
  \copyright\ 2017 Jason Underdown \\

  These flash cards are licensed under the:
  \begin{center}
    Creative Commons Attribution 4.0 \\
    International License \\
    \ccby
  \end{center}
  \url{https://creativecommons.org/licenses/by/4.0/}
\vspace*{\stretch{1}}
\end{flashcard}

\begin{flashcard}[Definition]{group}
  \vspace*{\stretch{1}}

  A \defn{group} consists of a set $G$, together with a rule for
  combining any two elements $g, h \in G$ to form another element of
  $G$ satisfying:
  \begin{enumerate}
  \item $\forall g,h,k \in G, (gh)k = g(hk)$
  \item $\exists e \in G \st \forall g \in G, eg=ge=g$
  \item $\forall g \in G, \exists g^{-1} \in G \st gg^{-1} = g^{-1}g = e$
  \end{enumerate}

  \vspace*{\stretch{1}}
\end{flashcard}

\begin{flashcard}[Definition]{subgroup}
\vspace*{\stretch{1}}

Let $G$ be a group. A subset $H$ of $G$ is a \defn{subgroup} if $H$ is
itself a group under the operation inherited from $G$.
\[
  H \le G
\]

\vspace*{\stretch{1}}
\end{flashcard}

\begin{flashcard}[Definition]{dihedral group $D_{2n}$}
  \vspace*{\stretch{1}}
  \[
    D_{2n} = \presentation{a,b}{a^n=1, b^2=1, b^{-1}ab=a^{-1}}
  \]

  \vspace*{\stretch{1}}
\end{flashcard}

\begin{flashcard}[Definition]{cyclic group $C_n$}
  \vspace*{\stretch{1}}

  \[
    C_n = \left\{ 1, a, a^2, \ldots, a^{n-1} \right\}
  \]
  \[
    C_n = \presentation{a}{a^n=1}
  \]

  \vspace*{\stretch{1}}
\end{flashcard}

\begin{flashcard}[Definition]{quaternion group $Q_8$}
  \vspace*{\stretch{1}}

  \[
    Q_8 = \presentation{a,b}{a^4 = 1, a^2 = b^2, b^{-1}ab = a^{-1}}
  \]

  \vspace*{\stretch{1}}
\end{flashcard}

\begin{flashcard}[Definition]{alternating group $A_n$}
  \vspace*{\stretch{1}}
  \[
    A_n = \left\{ g \in S_n : g \text{ is an even permutation} \right\}
  \]
  Recall that every permutation $g \in S_n$ can be expressed as a
  product of transpositions. An \defn{even} permutation has an even
  number of transpositions, and an \defn{odd} permutation has an odd
  number of transpositions.

  \vspace*{\stretch{1}}
\end{flashcard}

\begin{flashcard}[Definition]{direct product}
  \vspace*{\stretch{1}}

  Let $G$ and $H$ be groups, consider
  \[
    G \times H = \set{(g,h)}{g \in G \text{ and } h \in H}.
  \]
  Define a product operation on $G \times H$ by
  \[
    (g,h)(g',h') = (gg', hh').
  \]
  The group $G \times H$ is called the \defn{direct product} of $G$ and
  $H$.

  \vspace*{\stretch{1}}
\end{flashcard}

\begin{flashcard}[Definition]{homomorphism / isomorphism}
  \vspace*{\stretch{1}}

  If $G$ and $H$ are groups, then a \defn{homomorphism} from $G$ to
  $H$ is a map $\vartheta : G \to H$, which for all $g_1, g_2 \in G$
  satisfies:
  \[
    (g_1 g_2) \vartheta = (g_1 \vartheta)(g_2 \vartheta).
  \]
  If $\vartheta$ is also invertible, then $\vartheta$ is called an
  \defn{isomorphism}. \vspace*{\stretch{1}}
\end{flashcard}

\begin{flashcard}[Definition]{coset}
  \vspace*{\stretch{1}}
  Let $G$ be a group and $H$ a subgroup of $G$. For $x \in G$, the subset
  \[
    Hx = \set{hx}{h \in H}
  \]
  of $G$ is called a \defn{right coset} of $H$ in $G$. The distinct
  right cosets of $G$ partition $G$.
  \vspace*{\stretch{1}}
\end{flashcard}

%%%%%%%%%%%%%%%%%%%%%%%%%%%%%%%%%%%%%%%%%%%%%%%%%%%%%%%%%%%%%%%%%%%%%%%%%%%%%%%

\begin{flashcard}[Theorem]{Lagrange's theorem}
  \vspace*{\stretch{1}}
  If $G$ is a finite group and $H$ is a subgroup of $G$, then
  $\abs{H}$ divides $\abs{G}$.
  \vspace*{\stretch{1}}
\end{flashcard}

\begin{flashcard}[Definition]{index}
  \vspace*{\stretch{1}}

  Suppose $H$ is a subgroup of $G$. The number of distinct right
  cosets of $H$ in $G$ is written as $\abs{G : H}$. If $G$ is finite,
  then
  \[
    \abs{G : H} =\abs{G}/\abs{H}
  \]
  by Lagrange's theorem.

  \vspace*{\stretch{1}}
\end{flashcard}

\begin{flashcard}[Definition]{normal subgroup}
  \vspace*{\stretch{1}}
  A subgroup $N$ of a group $G$ is said to be a \defn{normal} subgroup
  of $G$ if $g^{-1}Ng = N$ for all $g \in G$, where
  \[
    g^{-1}Ng = \set{g^{-1}ng}{n \in N}.
  \]
  We indicate that $N$ is a normal subgroup of $G$ by writing:
  \[
    N \normal G.
  \]
  \vspace*{\stretch{1}}
\end{flashcard}

\begin{flashcard}[Definition]{factor group}
  \vspace*{\stretch{1}}

  If $N \normal G$, then define $G/N$ to be the set of right cosets of
  $N$ in $G$. This set is made into a group via the multiplication
  operation:
  \[
    (Ng)(Nh) = Ngh \qquad \forall g, h \in G.
  \]
  This operation makes $G/N$ into a group called the \defn{factor
    group} of $G$ by $N$.

  \vspace*{\stretch{1}}
\end{flashcard}

\begin{flashcard}[Definition]{simple group}
  \vspace*{\stretch{1}}

  A group $G$ is said to be \defn{simple} if $G \ne \{ 1 \}$ and the
  only normal subgroups of $G$ are $\{ 1 \}$ and $G$.
  \vspace*{\stretch{1}}
\end{flashcard}

\begin{flashcard}[Definition]{kernel / image}
  \vspace*{\stretch{1}}

  Let $G$ and $H$ be groups. Suppose that \[\theta : G \to H\] is a
  homomorphism then the \defn{kernel} of $\theta$ and \defn{image}
  of $\theta$ are defined to be:
  \begin{alignat*}{4}
    &\Ker \theta
    &&= \set{g \in G}{g\theta = 1} \quad
    && \Ker \theta &&\normal G \\
    &\Img \theta
    &&= \set{g\theta}{g \in G} \quad
    && \Img \theta &&\le H
  \end{alignat*}

  \vspace*{\stretch{1}}
\end{flashcard}

\begin{flashcard}[Theorem]{first isomorphism theorem}
  \vspace*{\stretch{1}}

  Suppose that $G$ and $H$ are groups and let $\theta: G \to H$ be a
  homomorphism. Then
  \[
    G / \Ker \theta \iso \Img \theta.
  \]
  An isomorphism is given by the function
  \[
    Kg \to g\theta \quad (g \in G)
  \]
  where $K = \Ker \theta$.
  \vspace*{\stretch{1}}
\end{flashcard}

\begin{flashcard}[Definition]{vector space}
  \vspace*{\stretch{1}}

  A \defn{vector space} over a field $F$ is a set $V$, equipped with
  addition and scalar multiplication satisfying:
  \begin{enumerate}
  \item $V$ is an abelian group under addition;
  \item $\forall \, u,v \in V$ and $\forall \, \lambda, \mu \in F$,
    \begin{enumerate}
    \item $\lambda(u + v) = \lambda u + \lambda v$
    \item $(\lambda + \mu)v = \lambda v + \mu v$
    \item $(\lambda \mu) v = \lambda (\mu v)$
    \item $1 v = v$
    \end{enumerate}
  \end{enumerate}

  \vspace*{\stretch{1}}
\end{flashcard}

\begin{flashcard}[Definition]{linear dependence / linear independence}
  \vspace*{\stretch{1}}

  We say that $v_1, \ldots, v_n$ are \defn{linearly dependent} if
  \[
    \lambda_1 v_1 + \cdots + \lambda_n v_n = 0
  \]
  for some $\lambda_1, \ldots, \lambda_n \in F$ not all zero,
  otherwise the vectors $v_1, \ldots, v_n$ are \defn{linearly
    independent}. \vspace*{\stretch{1}}
\end{flashcard}

\begin{flashcard}[Definition]{linear combination / span}
  \vspace*{\stretch{1}}

  Let $v_1, \ldots, v_n$ be vectors in a vector space $V$ over $F$. A
  vector $v$ in $V$ is a \defn{linear combination} of
  $v_1, \ldots, v_n$ if
  \[
    v = \lambda_1 v_1 + \cdots + \lambda_n v_n
  \]
  for some $\lambda_1, \ldots, \lambda_n \in F$.
  \vfill

  The vectors $v_1, \ldots, v_n$ \defn{span} $V$ if every vector in
  $V$ is a linear combination of $v_1, \ldots, v_n$.

  \vspace*{\stretch{1}}
\end{flashcard}

%%%%%%%%%%%%%%%%%%%%%%%%%%%%%%%%%%%%%%%%%%%%%%%%%%%%%%%%%%%%%%%%%%%%%%%%%%%%%%%

\begin{flashcard}[Definition]{basis}
  \vspace*{\stretch{1}}

  The vectors $v_1, \ldots , v_n \in V$ form a \defn{basis} of V if
  they
  \begin{enumerate}
  \item \emph{span} V, and are
  \item \emph{linearly independent}.
  \end{enumerate}

  \vspace*{\stretch{1}}
\end{flashcard}

\begin{flashcard}[Definition/Theorem]{subspace / conditions for a subspace}
  \vspace*{\stretch{1}}

  A \defn{subspace} of a vector space $V$ over $F$ is a subset of $V$
  which is itself a vector space under the operations inherited from
  $V$.
  \vfill

  A subset $U$ of a vector space $V$ is a subspace iff
  \begin{enumerate}
  \item $0\in U$;
  \item if $u,v \in U$ then $u+v \in U$;
  \item if $\lambda \in F$ and $u \in U$ then $\lambda u \in U$.
  \end{enumerate}

  \vspace*{\stretch{1}}
\end{flashcard}

\begin{flashcard}[Definition]{sum / direct sum}
  \vspace*{\stretch{1}}

  If $U_1, \ldots, U_r$ are subspaces of a vector space $V$, then
  define the \defn{sum of subspaces} to be
  \[
    U_1 + \cdots + U_r = \set{u_1 + \cdots + u_r}{u_i \in U_i \text{
        for } 1 \le i \le r}.
  \]
  \vfill

  If every element in $U_1 + \cdots + U_r$ can be written in a unique
  way as $u_1 + \cdots + u_r$ with $u_i \in U_i$ for $1 \le i \le r$,
  then the sum is called a \defn{direct sum} and is denoted:
  \[
    U_1 \oplus \cdots \oplus U_r
  \]

  \vspace*{\stretch{1}}
\end{flashcard}

\begin{flashcard}[Theorem]{conditions for a direct sum}
  \vspace*{\stretch{1}}

  Suppose that $V = U + W$, with $u_1, \ldots, u_r$ a basis of $U$ and
  $w_1, \ldots, w_s$ a basis of $W$, then the following three
  conditions are equivalent:
  \begin{enumerate}
  \item $V = U \oplus W$,
  \item $u_1, \ldots, u_r, w_1, \ldots, w_s$ is a basis of $V$,
  \item $U \cap W = \{ 0 \}$.
  \end{enumerate}

  \vspace*{\stretch{1}}
\end{flashcard}

\begin{flashcard}[Theorem]{direct sum of direct sums}
  \vspace*{\stretch{1}}

  Suppose $U, W, U_1, \ldots, U_a, W_1, \ldots, W_b$ are subspaces of
  the vector space $V$. If $V = U \oplus W$ and also
  \begin{align*}
    U &= U_1 \oplus \cdots \oplus U_a \\
    W &= W_1 \oplus \cdots \oplus W_b
  \end{align*}
  then
  \[
    V = U_1 \oplus \cdots \oplus U_a\oplus W_1 \oplus \cdots \oplus W_b.
  \]

  \vspace*{\stretch{1}}
\end{flashcard}

\begin{flashcard}[Definition]{external direct sum}
  \vspace*{\stretch{1}}

  Let $U_1, \ldots, U_r$ be vector spaces over $F$, and let
  \begin{align*}
    V &= \set{(u_1, \ldots, u_r)}{u_i \in U_i \text{ for } 1 \le i \le r},\\
    U'_i &= \set{(0, \ldots, u_i, \ldots, 0)}{u_i \in U_i}.
  \end{align*}
  Then $V = U'_1 \oplus \cdots \oplus U'_r$ is a vector space. Abusing
  notation slightly, we write
  \[
    V = U_1 \oplus \cdots \oplus U_r
  \]
  and call it the \defn{external direct sum} of $U_1, \ldots, U_r$.
  \vspace*{\stretch{1}}
\end{flashcard}

\begin{flashcard}[Definition]{linear transformation}
  \vspace*{\stretch{1}}

  Let $V$ and $W$ be vector spaces over $F$. A \defn{linear
    transformation} from $V$ to $W$ is a function
  \[
    \theta : V \to W
  \]
  which satisfies
  \begin{enumerate}
  \item $(u + v)\theta = u\theta + v\theta$ for all $u,v \in V$, and
  \item $(\lambda u)\theta = \lambda (v\theta)$ for all
    $\lambda \in F$ and $v \in V$.
  \end{enumerate}

  \vspace*{\stretch{1}}
\end{flashcard}

\begin{flashcard}[Theorem]{rank--nullity theorem}
  \vspace*{\stretch{1}}

  Suppose $V$ and $W$ are vector spaces and
  \[
    \theta : V \to W
  \]
  is a linear transformation, then
  \[
    \dim V = \dim(\Ker \theta) + \dim(\Img \theta)
  \]

  \vspace*{\stretch{1}}
\end{flashcard}

\begin{flashcard}[Theorem]{invertibility of linear transformations}
  \vspace*{\stretch{1}}

  Let $\theta$ be a linear transformation from $V$ to itself, then the
  following conditions are equivalent:
  \begin{enumerate}
  \item $\theta$ is invertible,
  \item $\Ker \theta = \{ 0 \}$,
  \item $\Img \theta  = V$.
  \end{enumerate}

  \vspace*{\stretch{1}}
\end{flashcard}

\begin{flashcard}[Definition]{endomorphism}
  \vspace*{\stretch{1}}

  A linear transformation from a vector space $V$ to itself is called
  an \defn{endomorphism} of $V$.

  \vspace*{\stretch{1}}
\end{flashcard}

%%%%%%%%%%%%%%%%%%%%%%%%%%%%%%%%%%%%%%%%%%%%%%%%%%%%%%%%%%%%%%%%%%%%%%%%%%%%%%%

\begin{flashcard}[Definition]{matrix of an endomorphism \\ $[\theta ]_{\B}$}
  \vspace*{\stretch{1}}

  Let $V$ be a vector space over $F$, and let $\theta$ be an
  endomorphism of $V$. Once a basis $\B = \{ v_1, \ldots, v_n \}$ for
  $V$ is chosen, then there are $n^2$ scalars
  $a_{ij} \in F \; (1 \le i,j \le n)$ such that for all $i$:
  \[
    v_i \theta = a_{i1}v_1 + \cdots + a_{in}v_n.
  \]
  The $n\times n$ matrix $(a_{ij})$ is called the \defn{matrix of
    $\theta$ relative to the basis $\B$}, and is denoted by
  $[\theta ]_{\B}$.

  \vspace*{\stretch{1}}
\end{flashcard}

\begin{flashcard}[Definition]{endomorphism algebra \\$\End(V)$}
  \vspace*{\stretch{1}}

  If $V$ is a vector space over $F$, then the set of endomorphisms of
  $V$ denoted $\End(V)$ form an algebra. Suppose
  $\theta, \phi \in \End(V)$ and $\lambda \in F$, then we
  define the functions $\theta + \phi$, $\theta \phi$ and
  $\lambda \theta$ from $V$ to $V$ by
  \begin{align*}
    v(\theta + \phi) &= v\theta + v\phi, \\
    v(\theta \phi) &= (v \theta) \phi, \\
    v(\lambda \theta) &= \lambda (v \theta),
  \end{align*}
  for all $v \in V$. Then $\theta + \phi$, $\theta \phi$ and
  $\lambda \theta$ are endomorphisms of $V$.

  \vspace*{\stretch{1}}
\end{flashcard}

\begin{flashcard}[Theorem]{$\theta \to [\theta]_{\B}$\\ is an algebra homomorphism}
  \vspace*{\stretch{1}}

  Suppose that $\B$ is a basis of the vector space $V$, and $\theta$
  and $\phi$ are endomorphisms of $V$, then
  \begin{align*}
    [\theta + \phi]_{\B} &= [\theta]_{\B} + [\phi]_{\B} \\
    [\theta \phi]_{\B} &= [\theta]_{\B} [\phi]_{\B} \\
    [\lambda \theta]_{\B} &= \lambda[\theta]_{\B}
  \end{align*}

  \vspace*{\stretch{1}}
\end{flashcard}

\begin{flashcard}[Definition]{change of basis matrix}
  \vspace*{\stretch{1}}

  Let $\B = \{v_1, \ldots, v_n \}$ be a basis of the vector space V,
  and let $\B' = \{v'_1, \ldots, v'_n \}$ be another basis of $V$.
  Then for $1 \le i \le n$,
  \[
    v'_i = t_{i1} v_1 + \cdots + t_{in} v_n
  \]
  for certain scalars $t_{ij}$. The $n\times n$ matrix $T=(t_{ij})$ is
  invertible and is called the \defn{change of basis matrix} from $\B$
  to $\B'$.

  \vspace*{\stretch{1}}
\end{flashcard}

\begin{flashcard}[Theorem]{change of basis}
  \vspace*{\stretch{1}}

  If $\B$ and $\B'$ are bases of $V$ and $\theta$ is an endomorphism
  of $V$, then
  \[
    [\theta]_{\B} = T^{-1}[\theta]_{\B'}T,
  \]
  where $T$ is the change of basis matrix from $\B$ to $\B'$.

  \vspace*{\stretch{1}}
\end{flashcard}

%%%%%%% Eigenvalue material should go here

\begin{flashcard}[Proposition]{direct sums induce projections}
  \vspace*{\stretch{1}}

  Suppose that $V = U \oplus W$. Define $\pi : V \to V$ by
  \[
    (u+w)\pi = u \qquad \text{ for all } u\in U, w \in W.
  \]
  Then $\pi$ is an endomorphism of $V$. Further
  \[
    \Img \pi = U, \quad \Ker \pi = W, \, \text{ and } \, \pi^2 = \pi.
  \]

  \vspace*{\stretch{1}}
\end{flashcard}

\begin{flashcard}[Definition]{projection}
  \vspace*{\stretch{1}}

  An endomorphism $\pi$ of a vector space $V$ satisfying $\pi^2 = \pi$
  is called a \defn{projection} of $V$.

  \vspace*{\stretch{1}}
\end{flashcard}

\begin{flashcard}[Proposition]{projections induce direct sum decomposition}
  \vspace*{\stretch{1}}

  Suppose that $\pi$ is a projection of a vector space $V$. Then
  \[
    V = \Img \pi \oplus \Ker \pi
  \]

  \vspace*{\stretch{1}}
\end{flashcard}

\begin{flashcard}[Definition]{representation of a group / degree}
  \vspace*{\stretch{1}}

  A \defn{representation} of $G$ over $F$ is a homomorphism
  \[
    \rho : G \to GL(n, F) \quad \text{ for some } n.
  \]
  The \defn{degree} of $\rho$ is the integer $n$.

  \vspace*{\stretch{1}}
\end{flashcard}

\begin{flashcard}[Definition]{equivalent representations}
  \vspace*{\stretch{1}}

  Let $\rho : G \to GL(m, F)$ and $\sigma : G \to GL(n, F)$ be
  representations of $G$ over $F$. We say that $\rho$ is
  \defn{equivalent} to $\sigma$ if $n=m$ and there exists an
  invertible $n\times n$ matrix $T$ such that for all $g \in G$,
  \[
    g\sigma = T^{-1}(g\rho)T.
  \]
  Equivalence of representations is an equivalence relation.

  \vspace*{\stretch{1}}
\end{flashcard}

\begin{flashcard}[Definition]{trivial representation}
  \vspace*{\stretch{1}}

  The representation $\rho: G \to \GLnF{1}{F}$ defined by
  \[
    g \rho = (1)\qquad \text{ for all } g \in G,
  \]
  is called the \defn{trivial representation} of $G$.

  \vspace*{\stretch{1}}
\end{flashcard}

\begin{flashcard}[Definition]{faithful representation}
  \vspace*{\stretch{1}}

  A representation $\rho : G \to \GLnF{n}{F}$ is said to be
  \defn{faithful} if $\Ker{\rho} = \{ 1 \}$; that is, if the identity
  element of $G$ is the only element $g$ for which $g\rho = I_n$.

  \vspace*{\stretch{1}}
\end{flashcard}

\begin{flashcard}[Proposition]{$\rho$ faithful $\Leftrightarrow$
    $\Img \rho \iso G$ }
  \vspace*{\stretch{1}}

  A representation $\rho$ of a finite group is faithful if and only if
  $\Img \rho$ is isomorphic to $G$.

  \vspace*{\stretch{1}}
\end{flashcard}

\begin{flashcard}[Definition]{$FG$-module}
  \vspace*{\stretch{1}}

  Let $V$ be a vector space over $F$ and $G$ a group. Then $V$ is an
  \defn{$FG$-module} if a multiplication $vg$ is defined and satisfies,
  for all $u,v \in V, \lambda \in F$ and $g, h \in G$:
  \begin{enumerate}
  \item $vg \in V$
  \item $v(gh) = (vg)h$
  \item $v1 = v$
  \item $(\lambda v)g = \lambda(vg)$
  \item $(u+v)g = ug + vg$
  \end{enumerate}

  \vspace*{\stretch{1}}
\end{flashcard}

\begin{flashcard}[Definition]{matrix of an endomorphism}
  \vspace*{\stretch{1}}

  Let $V$ be an $FG$-module, and let $\B$ be a basis of $V$. For each
  $g\in G$, let
  \[
    [g]_{\B}
  \]
  denote the matrix of the endomorphism $v \mapsto vg$ of $V$,
  relative to the basis $\B$.

  \vspace*{\stretch{1}}
\end{flashcard}

\begin{flashcard}[Theorem 4.4]{representations induce $FG$-modules \\
    and vice versa}
  \vspace*{\stretch{1}}

  \begin{enumerate}
  \item If $\rho: G \to \GLnF{n}{F}$ is a representation of $G$ over
    $F$, $V=F^n$, then $V$ becomes an $FG$-module by defining the
    multiplication to be $vg = v(g\rho)$. Moreover there exists a
    basis $\B$ of $V$ such that $g\rho = [g]_{\B}$.

  \item If $V$ is an $FG$-module and $\B$ a basis of $V$, then
    $\rho: g \mapsto [g]_{\B}$ is  a representation of $G$ over $F$.
  \end{enumerate}

  \vspace*{\stretch{1}}
\end{flashcard}

\begin{flashcard}[Proposition 4.6]{defining the action of $G$ on a basis of $V$\\
  induces an $FG$-module}
  \vspace*{\stretch{1}}

  Let $\B=\{v_1, \ldots , v_n\}$ be a basis for a vector space $V$
  over $F$. If $v_i g$ is defined for all $v_i \in \B$ and for all
  $g\in G$ and satisfies $\forall \, g, h\in G$, and
  $\forall \, \lambda_1, \ldots, \lambda_n \in F$:
  \begin{enumerate}
  \item $v_i g \in V$
  \item $v_i(gh) = (v_i g)h$
  \item $v_i 1 = v_i$
  \item $(\lambda_1 v_1 + \cdots + \lambda_n v_n)g =
    \lambda_1(v_1 g) + \cdots + \lambda_n (v_n g)$
  \end{enumerate}
  Then $V$ is an $FG$-module.

  \vspace*{\stretch{1}}
\end{flashcard}

\begin{flashcard}[Definition]{the trivial $FG$-module}
  \vspace*{\stretch{1}}

  The \defn{trivial} $FG$-module is the 1-dimensional vector space $V$
  over $F$ with
  \[
    vg = v \quad \text{ for all } \quad v\in V, \: g\in G.
  \]

  \vspace*{\stretch{1}}
\end{flashcard}

\begin{flashcard}[Definition]{faithful $FG$-module}
  \vspace*{\stretch{1}}

  An $FG$-module $V$ is \defn{faithful} if the identity element of $G$
  is the only element of $G$ for which
  \[
    vg = v \quad \text{for all} \quad v\in V.
  \]

  \vspace*{\stretch{1}}
\end{flashcard}

\begin{flashcard}[Definition]{permutation module}
  \vspace*{\stretch{1}}

  Let $G$ be a subgroup of $S_n$. The $FG$-module $V$ with basis
  $v_1, \ldots, v_n$ such that
  \[
    v_i g = v_{ig} \quad \text{for all } i, \text{ and all } g\in G,
  \]
  is called the \defn{permutation module} for $G$ over $F$. We call
  $v_1, \ldots, v_n$ the \defn{natural basis} of $V$.

  \vspace*{\stretch{1}}
\end{flashcard}



%%%%%%%%%%%%%%%%%%%%%%%%%%%%%%%%%%%%%%%%%%%%%%%%%%%%%%%%%%%%%%%%%%%%%%%%%%%%%%%
\end{document}
